\documentclass[12pt,a4paper]{article}
\usepackage[utf8]{inputenc}
\usepackage{polski}
\usepackage{geometry}
\geometry{margin=2.5cm}

\let\stdsection\section
\renewcommand\section{\clearpage\stdsection}

\title{Regulamin drużyny hokejowej Papugi}
\date{\today}

\begin{document}

\maketitle
\tableofcontents

\section{Postanowienia ogólne}
\begin{enumerate}
  \item Drużyna amatorska hokeja na lodzie nosi nazwę \textbf{Papugi}.
  \item Drużyna nie prowadzi działalności gospodarczej i nie działa dla zysku. Jej celem jest umożliwienie członkom wspólnej gry w hokeja, integracji oraz rozwoju umiejętności sportowych.
  \item Drużyna działa na terenie Katowic i okolic, w szczególności korzystając z lodowiska Satelita.
  \item Symbolem drużyny jest nazwa i logo \textbf{Papugi}, wykorzystywane wyłącznie do celów sportowych i promocyjnych.
  \item Regulamin określa zasady funkcjonowania drużyny, prawa i obowiązki zawodników oraz zasady organizacyjne.
\end{enumerate}

\section{Członkostwo}
\begin{enumerate}
  \item Członkami drużyny mogą być wyłącznie osoby pełnoletnie. Osoby niepełnoletnie mogą zostać członkami drużyny jedynie po przedstawieniu dokumentu zatwierdzającego udział w zespole, podpisanego przez rodzica lub opiekuna prawnego.
  \item W drużynie Papugi wyróżnia się trzy typy uczestników, różniące się zakresem obowiązków i poziomem zaangażowania.
\end{enumerate}

\subsection{Gość (Jajko)}
\textbf{Plusy:}
\begin{itemize}
    \item dostęp do wewnętrznej grupy dla gości,
    \item możliwość uczestniczenia w treningach jako gość,
    \item szansa na poznanie drużyny bez zobowiązań i późniejsze zostanie pełnoprawną Papugą.
\end{itemize}
\textbf{Minusy:}
\begin{itemize}
    \item wyższa opłata za treningi (60 zł),
    \item brak gwarancji miejsca na treningu i sparingu,
    \item brak prawa głosu i udziału w decyzjach organizacyjnych.
\end{itemize}

\subsection{Zawodnik (Papuga)}
\textbf{Plusy:}
\begin{itemize}
    \item niższe koszty treningów dzięki stałym składkom,
    \item dostęp do wewnętrznej grupy drużyny,
    \item gwarancja miejsca na treningach i sparingach,
    \item możliwość korzystania ze strojów drużynowych,
    \item wsparcie trenera oraz bardziej doświadczonych zawodników w rozwoju umiejętności,
    \item drużynowy networking i integracja,
    \item prawo głosu i wpływ na kierunek rozwoju drużyny.
\end{itemize}

\subsection{Członek zarządu (Stara Papuga)}
\textbf{Plusy:}
\begin{itemize}
    \item największy wpływ na kierunek i rozwój drużyny,
    \item odpowiedzialność za podejmowanie decyzji organizacyjnych i finansowych,
    \item reprezentowanie drużyny wobec instytucji zewnętrznych.
\end{itemize}

\subsection{Obowiązki wspólne wszystkich uczestników}
\begin{enumerate}
    \item Przestrzeganie zasad fair play na lodzie i poza nim.
    \item Dbanie o własny sprzęt sportowy oraz wspólny sprzęt drużynowy.
    \item Zachowanie kultury osobistej wobec członków drużyny, trenerów, przeciwników i sędziów.
    \item Regularne informowanie o swojej obecności bądź nieobecności na treningach i sparingach.
    \item Terminowe opłacanie składek (dotyczy zawodników).
    \item Wspieranie pozytywnej atmosfery i integracji w drużynie.
\end{enumerate}

\subsection{Zmiana statusu uczestnika}
Aby przejść z roli gościa (Jajko) do roli zawodnika (Papuga), uczestnik musi odbyć co najmniej pięć treningów, a następnie zgłosić swoją wolę zostania zawodnikiem do zarządu drużyny.

\subsection{Utrata członkostwa}
Członkostwo w drużynie Papugi można utracić w przypadku:
\begin{enumerate}
    \item długotrwałego nieopłacania składek,
    \item rażącego naruszania zasad fair play,
    \item powtarzających się zachowań agresywnych lub niekulturalnych wobec członków drużyny, trenerów, sędziów bądź przeciwników,
    \item niszczenia lub nieodpowiedzialnego korzystania ze sprzętu drużynowego,
    \item nieusprawiedliwionej, długotrwałej nieobecności na treningach i meczach,
    \item działania na szkodę drużyny.
\end{enumerate}
Decyzję o utracie członkostwa podejmuje zarząd drużyny. Opłacone składki po utracie członkostwa nie podlegają zwrotowi.

\section{Zasady treningów i sparingów}
Wszelkie kwestie organizacyjne związane z treningami i sparingami (terminy, dojazdy, przeciwnicy) ogłaszane są w oficjalnym kanale komunikacji drużyny.

\subsection{Treningi}
\begin{enumerate}
  \item Treningi odbywają się zgodnie z harmonogramem ustalonym przez drużynę.
  \item Udział w treningach jest oczekiwany, ponieważ pozwala to na lepszą współpracę w zespole, płynniejszą rozgrywkę oraz skuteczniejsze przygotowanie do sparingów. W przypadku nieobecności należy poinformować o tym wcześniej w grupie drużynowej.
  \item Punktualne przybycie na trening jest zalecane. Spóźnienia zakłócają przebieg zajęć i utrudniają organizację.
  \item Zawodnik powinien stawiać się na trening w pełnym, sprawnym sprzęcie ochronnym. Dodatkowo ma obowiązek posiadać przy sobie białą i czarną bluzę hokejową, co umożliwia sprawny podział na drużyny podczas zajęć.

\end{enumerate}

\subsection{Sparingi}
\begin{enumerate}
  \item Sparingi organizowane są w celu podnoszenia umiejętności drużyny i integracji zawodników.
  \item W sparingach biorą udział zawodnicy, którzy wykazują systematyczną obecność na treningach i opłacają składki w terminie.
  \item Skład na sparingi ustala zarząd, trener i kapitan drużyny, biorąc pod uwagę frekwencję na treningach, zaangażowanie i aktualne potrzeby zespołu.
  \item W sparingach mogą brać udział goście (\textit{Jajka}), jeżeli pozwala na to liczba miejsc i wyrazi zgodę zarząd.
  \item Koszty związane z organizacją sparingu (wynajem lodowiska, opłaty sędziowskie) mogą być dzielone pomiędzy uczestników, jeśli nie są pokryte ze składek drużyny.
  \item W przypadku braku możliwości rozegrania sparingu w ustalonym terminie, zarząd może wyznaczyć nową datę lub odwołać wydarzenie.
\end{enumerate}

\section{Fair play i bezpieczeństwo}
\begin{enumerate}
  \item Każdy uczestnik zobowiązany jest do przestrzegania zasad fair play podczas treningów, sparingów oraz poza lodowiskiem.
  \item Niedopuszczalne są zachowania agresywne wobec członków drużyny, przeciwników, trenerów, sędziów i organizatorów.
  \item Wszyscy zawodnicy muszą korzystać z pełnego, sprawnego sprzętu ochronnego: kasku, ochraniaczy, rękawic i łyżew.
  \item Każdy uczestnik odpowiada za własny sprzęt sportowy, a za sprzęt drużynowy odpowiedzialność ponoszą wspólnie wszyscy członkowie.
  \item Zawodnicy zobowiązani są do dbania o bezpieczeństwo swoje i innych, w szczególności poprzez unikanie ryzykownych zachowań na lodzie.
  \item W razie kontuzji lub urazu zawodnik powinien niezwłocznie zgłosić to trenerowi lub kapitanowi i w razie potrzeby skorzystać z pomocy medycznej.
  \item Uczestnicy zobowiązani są do kulturalnego zachowania wobec publiczności oraz poszanowania obiektów sportowych, w których odbywają się zajęcia.
\end{enumerate}

\section{Składki i finanse drużyny}
\begin{enumerate}
  \item Sezon trwa od września 2025 roku do kwietnia 2026 roku, co daje 8 miesięcy gry.
  \item W składce uwzględniony jest jeden trening z trenerem tygodniowo. Dodatkowo drużyna organizuje jeden dodatkowy trening bez trenera, który nie jest wliczony w składkę.
  \item Wysokość składki wynosi 50 zł za każdy trening z trenerem, pomnożone przez ilość piątków (dni treningowych) w danym miesiącu.
  \item Składka płatna jest gotówką skarbnikowi podczas pierwszego treningu w danym miesiącu, w którym zawodnik bierze udział.
  \item Treningi składkowe z trenerem odbywają się na lodowisku Satelita w Katowicach w piątki o 21:00. W przypadku odwołania treningu z przyczyn niezależnych od drużyny, organizatorzy podejmą próbę zorganizowania go w innym terminie i miejscu w tym samym tygodniu. Jeśli nie będzie to możliwe, składka za ten trening nie przepada.
  \item Cena za trening: zawodnik drużyny -- 50 zł, gość -- 60 zł.
  \item Cena przewidziana jest dla 20 stałych zawodników, wraz ze wzrostem ilości zawodników cena składki będzie spadać lub oferowane będą inne benefity.
  \item W przypadku nieobecności na treningu składka za dany trening przepada.
  \item W sytuacji długotrwałej niedyspozycji (np. kontuzji) zawodnik powinien zgłosić się do skarbnika w celu indywidualnego ustalenia zasad płatności.
\end{enumerate}

\subsection{Tabela składek za sezon 2025/2026}
\begin{center}
\begin{tabular}{ |c|c|c| } 
 \hline
 Miesiąc & Ilość tygodni & Przewidywana kwota składki (zł) \\ 
 \hline
 Wrzesień 2025 & 4 & 200 \\ 
 Październik 2025 & 5 & 250 \\ 
 Listopad 2025 & 4 & 200 \\ 
 Grudzień 2025 & 4 & 200 \\ 
 Styczeń 2026 & 5 & 250 \\ 
 Luty 2026 & 4 & 200 \\ 
 Marzec 2026 & 4 & 200 \\ 
 Kwiecień 2026 & 4 & 200 \\ 
 \hline
 \textbf{Razem} & 34 & 1700 \\ 
 \hline
\end{tabular}
\end{center}

\section{Organizacja i komunikacja}
\begin{enumerate}
  \item Wszystkie decyzje organizacyjne dotyczące drużyny podejmowane są przez zarząd drużyny.
  \item W sprawach finansowych odpowiedzialność ponosi skarbnik drużyny.
  \item Głównym kanałem komunikacji drużyny są grupy na WhatsApp, w których publikowane są informacje o treningach, sparingach i innych wydarzeniach.
  \item Każdy zawodnik ma obowiązek śledzić komunikaty publikowane w grupach i reagować na nie w rozsądnym terminie.
  \item Kapitan, skarbnik oraz inni wyznaczeni członkowie odpowiadają za przekazywanie informacji i dbanie o sprawny przepływ komunikatów w drużynie.
\end{enumerate}

\section{Postanowienia końcowe}
\begin{enumerate}
  \item Regulamin wchodzi w życie z dniem jego przyjęcia przez zarząd drużyny.
  \item Wszelkie zmiany w regulaminie mogą być wprowadzane wyłącznie decyzją zarządu drużyny.
  \item Sprawy nieuregulowane w regulaminie rozstrzygane są przez zarząd drużyny.
  \item Regulamin ma charakter wewnętrzny i obowiązuje wszystkich uczestników drużyny Papugi.
\end{enumerate}

\end{document}
