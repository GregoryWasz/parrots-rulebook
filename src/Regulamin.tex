\documentclass[12pt,a4paper]{article}
\usepackage[utf8]{inputenc}
\usepackage{polski}
\usepackage{geometry}
\geometry{margin=2.5cm}

\title{Regulamin drużyny hokejowej Papugi}
\date{\today}

\begin{document}

\maketitle

\section{Postanowienia ogólne}
1. Drużyna amatorska hokeja na lodzie nosi nazwę \textbf{Papugi}.\\
2. Drużyna nie prowadzi działalności gospodarczej i nie działa dla zysku. Jej celem jest umożliwienie członkom wspólnej gry w hokeja, integracji oraz rozwoju umiejętności sportowych.\\
3. Drużyna działa na terenie Katowic i okolic, w szczególności korzystając z lodowiska Satelita.\\
4. Symbolem drużyny jest nazwa i logo \textbf{Papugi}, wykorzystywane wyłącznie do celów sportowych i promocyjnych.\\
5. Regulamin określa zasady funkcjonowania drużyny, prawa i obowiązki zawodników oraz zasady organizacyjne.

\section{Członkostwo}
1. Członkami drużyny mogą być wyłącznie osoby pełnoletnie. Osoby niepełnoletnie mogą zostać członkami drużyny jedynie po przedstawieniu dokumentu zatwierdzającego udział w zespole, podpisanego przez rodzica lub opiekuna prawnego.\\
2. W drużynie Papugi wyróżnia się trzy typy uczestników, różniące się zakresem obowiązków i poziomem zaangażowania. \\

\subsection{Gość (Jajko)}
\textbf{Plusy:}
\begin{itemize}
    \item dostęp do wewnętrznej grupy dla gości,
    \item możliwość uczestniczenia w treningach jako gość,
    \item szansa na poznanie drużyny bez zobowiązań i późniejsze zostanie pełnoprawną Papugą.
\end{itemize}
\textbf{Minusy:}
\begin{itemize}
    \item wyższa opłata za treningi (60 zł),
    \item brak gwarancji miejsca na treningu i meczu,
    \item brak prawa głosu i udziału w decyzjach organizacyjnych.
\end{itemize}

\subsection{Zawodnik (Papuga)}
\textbf{Plusy:}
\begin{itemize}
    \item niższe koszty treningów dzięki stałym składkom,
    \item dostęp do wewnętrznej grupy drużyny,
    \item gwarancja miejsca na treningach i meczach,
    \item możliwość korzystania ze strojów drużynowych,
    \item wsparcie trenera oraz bardziej doświadczonych zawodników w rozwoju umiejętności,
    \item drużynowy networking i integracja,
    \item prawo głosu i wpływ na kierunek rozwoju drużyny.
\end{itemize}

\subsection{Członek zarządu (Papuga Janusz)}
\textbf{Plusy:}
\begin{itemize}
    \item największy wpływ na kierunek i rozwój drużyny,
    \item odpowiedzialność za podejmowanie decyzji organizacyjnych i finansowych,
    \item reprezentowanie drużyny wobec instytucji zewnętrznych.
\end{itemize}

\subsection{Obowiązki wspólne wszystkich uczestników}
\begin{enumerate}
    \item Przestrzeganie zasad fair play na lodzie i poza nim.
    \item Dbanie o własny sprzęt sportowy oraz wspólny sprzęt drużynowy.
    \item Zachowanie kultury osobistej wobec członków drużyny, trenerów, przeciwników i sędziów.
    \item Regularne informowanie o swojej obecności bądź nieobecności na treningach i meczach.
    \item Terminowe opłacanie składek (dotyczy zawodników).
    \item Wspieranie pozytywnej atmosfery i integracji w drużynie.
\end{enumerate}
\subsection{Zmiana statusu uczestnika}
Aby przejść z roli gościa (Jajko) do roli zawodnika (Papuga), uczestnik musi odbyć co najmniej pięć treningów, a następnie zgłosić swoją wolę zostania zawodnikiem do zarządu drużyny.

\section{Zasady treningów i meczów}
1. Treningi odbywają się zgodnie z harmonogramem ustalonym przez drużynę.\\
2. Udział w treningach jest oczekiwany, ponieważ pozwala to na lepszą współpracę w zespole, płynniejszą rozgrywkę oraz skuteczniejsze przygotowanie do meczów. W przypadku nieobecności należy poinformować o tym wcześniej w grupie drużynowej.\\
3. Punktualne przybycie na trening jest zalecane. Spóźnienia zakłócają przebieg zajęć i utrudniają organizację.\\
4. Zawodnik powinien stawiać się na trening w pełnym, sprawnym sprzęcie ochronnym. Dodatkowo ma obowiązek posiadać przy sobie białą i czarną bluzę hokejową, co umożliwia sprawny podział na drużyny podczas zajęć.\\
5. W meczach biorą udział zawodnicy, którzy wykazują systematyczną obecność na treningach i opłacają składki w terminie.\\
6. Skład drużyny na mecze ustalany jest przez trenera lub kapitana, z uwzględnieniem frekwencji, zaangażowania oraz poziomu sportowego.\\
7. W przypadku sparingów lub meczów towarzyskich możliwe jest dopuszczenie do gry gości (\textit{Jajek}), o ile pozwala na to liczba miejsc i decyzja organizatorów.\\
8. Wszelkie kwestie organizacyjne związane z treningami i meczami (terminy, dojazdy, przeciwnicy) ogłaszane są w oficjalnym kanale komunikacji drużyny.

\subsection{Sparingi}
1. Sparingi organizowane są w celu podnoszenia umiejętności drużyny i integracji zawodników.\\
2. Skład na sparingi ustala zarząd drużyny, biorąc pod uwagę frekwencję na treningach, zaangażowanie i aktualne potrzeby zespołu.\\
3. W sparingach mogą brać udział goście (\textit{Jajka}), jeżeli pozwala na to liczba miejsc i wyrazi zgodę zarząd.\\
4. Koszty związane z organizacją sparingu (wynajem lodowiska, opłaty sędziowskie) mogą być dzielone pomiędzy uczestników, jeśli nie są pokryte ze składek drużyny.\\
5. W przypadku braku możliwości rozegrania sparingu w ustalonym terminie, zarząd może wyznaczyć nową datę lub odwołać wydarzenie.

\section{Fair play i bezpieczeństwo}
1. Każdy uczestnik zobowiązany jest do przestrzegania zasad fair play podczas treningów, meczów oraz poza lodowiskiem.\\
2. Niedopuszczalne są zachowania agresywne wobec członków drużyny, przeciwników, trenerów, sędziów i organizatorów.\\
3. Wszyscy zawodnicy muszą korzystać z pełnego, sprawnego sprzętu ochronnego: kasku, ochraniaczy, rękawic i łyżew.\\
4. Każdy uczestnik odpowiada za własny sprzęt sportowy, a za sprzęt drużynowy odpowiedzialność ponoszą wspólnie wszyscy członkowie.\\
5. Zawodnicy zobowiązani są do dbania o bezpieczeństwo swoje i innych, w szczególności poprzez unikanie ryzykownych zachowań na lodzie.\\
6. W razie kontuzji lub urazu zawodnik powinien niezwłocznie zgłosić to trenerowi lub kapitanowi i w razie potrzeby skorzystać z pomocy medycznej.\\
7. Uczestnicy zobowiązani są do kulturalnego zachowania wobec publiczności oraz poszanowania obiektów sportowych, w których odbywają się zajęcia.

\section{Składki i finanse drużyny}
1. Sezon trwa od września 2025 roku do kwietnia 2026 roku, co daje 8 miesięcy gry.\\
2. W składce uwzględniony jest jeden trening z trenerem tygodniowo. Dodatkowo drużyna organizuje jeden dodatkowy trening bez trenera, który nie jest wliczony w składkę.\\
3. Wysokość składki wynosi 50 zł za każdy trening z trenerem, pomnożone przez liczbę tygodni w danym miesiącu.\\
4. Składka płatna jest gotówką skarbnikowi podczas pierwszego treningu w danym miesiącu, w którym zawodnik bierze udział.\\
5. Treningi składkowe z trenerem odbywają się na lodowisku Satelita w Katowicach w piątki o 21:00. W przypadku odwołania treningu z przyczyn niezależnych od drużyny, organizatorzy podejmą próbę zorganizowania go w innym terminie i miejscu w tym samym tygodniu. Jeśli nie będzie to możliwe, składka za ten trening nie przepada.\\
6. Cena za trening: zawodnik drużyny -- 50 zł, gość -- 60 zł.\\
7. W przypadku nieobecności na treningu składka za dany trening przepada.\\
8. W sytuacji długotrwałej niedyspozycji (np. kontuzji) zawodnik powinien zgłosić się do skarbnika w celu indywidualnego ustalenia zasad płatności.

\section{Organizacja i komunikacja}
1. Wszystkie decyzje organizacyjne dotyczące drużyny podejmowane są przez zarząd drużyny.\\
2. W sprawach finansowych odpowiedzialność ponosi skarbnik drużyny.\\
3. Głównym kanałem komunikacji drużyny są grupy na WhatsApp, w których publikowane są informacje o treningach, meczach i innych wydarzeniach.\\
4. Każdy zawodnik ma obowiązek śledzić komunikaty publikowane w grupach i reagować na nie w rozsądnym terminie.\\
5. Kapitan, skarbnik oraz inni wyznaczeni członkowie odpowiadają za przekazywanie informacji i dbanie o sprawny przepływ komunikatów w drużynie.

\section{Postanowienia końcowe}
1. Regulamin wchodzi w życie z dniem jego przyjęcia przez członków drużyny.\\
2. Wszelkie zmiany w regulaminie mogą być wprowadzane wyłącznie decyzją zarządu drużyny.\\
3. Sprawy nieuregulowane w regulaminie rozstrzygane są przez zarząd drużyny.\\
4. Regulamin ma charakter wewnętrzny i obowiązuje wszystkich uczestników drużyny Papugi.

\end{document}
