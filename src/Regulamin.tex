\documentclass{article}

\usepackage[polish]{babel}   % obsługa polskiego
\usepackage[utf8]{inputenc}  % kodowanie UTF-8
\usepackage[T1]{fontenc}     % kodowanie fontów, żeby polskie znaki były poprawne

\title{Zespół}
\date{\today}

\begin{document}

% TUTAJ WPISUJE
\maketitle

\section{Regulamin}

Regulamin polskiego zespołu hokeja na lodzie określa zasady uczestnictwa w treningach, meczach i turniejach. Zawodnicy zobowiązani są do przestrzegania zasad fair play oraz do dbania o sprzęt i bezpieczeństwo na lodzie.

\subsection{Składki}

Składka na 4 tygodnie obejmuje 4 kolejne treningi z trenerem w kolejnym miesiącu. Cena wynosi 200 zł 00 groszy. Zespół stara się odbyć jeden trening z trenerem w tygodniu, zazwyczaj w czwartek na lodowisku Satelita w Katowicach.  

Jeżeli nie uda się przeprowadzić treningu, zostaje on przeniesiony na następny tydzień. Nieobecność na opłaconym treningu skutkuje brakiem jakiejkolwiek rekompensaty.  

W celu utrzymania porządku, wszyscy zawodnicy utrzymywani są w tym samym okresie rozliczeniowym. Jedynie nowo przyjęte papugi mogą jednorazowo zacząć składkę w środku orkesu rozliczeniowego.

\section{Zawodnik}

Każdy zawodnik zespołu ma przypisaną rolę. Zawodnicy biorą udział w treningach, meczach. Istotna jest kondycja fizyczna, technika gry i współpraca w drużynie.

\subsection{Strój}

Zawodnik to członek drużyny, który bierze udział w treningach i meczach. Musi mieć cały strój i sprzęt hokejowy, w tym dwie koszulki treningowe – jedną czarną i jedną białą.  


\subsection{Zachowanie}

Jeśli coś mu przeszkadza albo ma problem, powinien powiedzieć o tym kapitanowi lub trenerowi, ale podczas skarg zaproponować własny pomysł, jak rozwiązać problem, zamiast tylko narzekać.

\section{Kapitan}

Kapitan drużyny pełni funkcję lidera zarówno na lodzie, jak i poza nim. Jest odpowiedzialny za komunikację z trenerem, motywowanie zespołu oraz reprezentowanie drużyny podczas spotkań oficjalnych. Po skończonym meczu ma obowiązek analizy gry wraz z trenerem oraz przekazania informacji zwrotnej zawodnikom na temat tego co działa dobrze i źle. Musi mieć zaplanowaną strategię oraz taktykę na nadchodzące spotkania i treningi.

\section{Poradniki}

Poradniki obejmują wskazówki dotyczące techniki strzałów, jazdy na łyżwach, taktyki zespołowej oraz przygotowania fizycznego. Są przydatne zarówno dla początkujących, jak i doświadczonych zawodników.

\subsection{Spalony}

Spalony (ang. offside) występuje, gdy zawodnik znajdzie się w tercji obronnej przeciwnika zanim zostanie wbity tam krążek przez jego drużynę. Wówczas gra wraca do tercji środkowej (neutralnej). Obecnie stosuje się tzw. „tag-up offside”, czyli odłożony spalony. Oznacza to, że gracz znajdujący się na pozycji spalonej może z niej uciec, czyli wycofać się do strefy neutralnej, a następnie ponownie wjechać do strefy obronnej rywala – wówczas gra nie zostaje przerwana przez sędziego.

\subsection{Uwolnienie}

Uwolnienie (ang. icing) występuje, gdy gracz wybije krążek z własnej połowy (własnej tercji lub własnej połowy) na połowę przeciwnika w taki sposób, że krążek minie linię bramkową po drugiej stronie tafli oraz gdy żaden z zawodników nie będzie miał możliwości przejęcia go. Wówczas gra zostaje przerwana i następuje wznowienie w tercji obronnej drużyny dokonującej wystrzelenia krążka. Jest to tzw. zabronione uwolnienie, gdyż dokonywane jest przy równej liczbie graczy obu drużyn na lodzie. Uwolnienie dozwolone jest dopuszczalne dla drużyny grającej w liczebnym osłabieniu. Wówczas możliwość wystrzelenia krążka na drugi koniec tafli jest przywilejem w grze obronnej dla drużyny osłabionej. Ponadto zabronione uwolnienie jest bezskuteczne, gdy krążek trafi do bramki. Od sezonu 2008/09 IIHF wprowadziła nową zasadę, w myśl której drużyna przeciwna po dokonaniu zabronionego uwolnienia nie może dokonać zmiany zawodników i do wznowienia muszą przystąpić gracze, którzy przebywali na lodzie w momencie dokonania uwolnienia.


\section{Roadmapa}

Roadmapa zespołu opisuje plan sezonu, harmonogram treningów, ważne turnieje oraz cele drużyny. Umożliwia monitorowanie postępów i przygotowanie strategii na nadchodzące mecze.


\subsection{Wprowadzenie strojów treningowych (TBA)}

\subsection{Wprowadzenie składek (Plan Wrzesień)}

Wszyscy członkowie drużyny będą wpłacać składki, które będą przeznaczone na sprzęt, wynajem lodowiska i inne potrzeby drużyny. Składki mają zapewnić równy dostęp do zasobów dla każdego zawodnika.

\subsection{Wprowadzenie dwóch regularnych treningów (Plan Wrzesień)}

Drużyna będzie miała dwa typy treningów tygodniowo:
\begin{itemize}
    \item Jeden trening z trenerem – skoncentrowany na taktyce, technice i indywidualnym rozwoju zawodników.
    \item Jeden trening bez trenera – samodzielny lub zespołowy, pozwalający na samodzielną pracę nad umiejętnościami oraz integrację drużyny.
\end{itemize}

\subsection{Ustalenie Alt Kapitanów (Plan Wrzesień)}

\subsection{Stworzenie okrzyku (Plan Wrzesień)}

\subsection{Integracja na rozpoczęcie sezonu (Plan Wrzesień)}

\subsection{Spraing z inną drużyną (Plan Wrzesień)}

\section{Marketing}

Sekcja marketingu odpowiada za promocję zespołu, budowanie wizerunku, kontakty z mediami oraz pozyskiwanie sponsorów. Obejmuje działania w social mediach, organizację wydarzeń dla fanów i współpracę z lokalnymi społecznościami, aby zwiększyć zainteresowanie hokejem na lodzie.

\subsection{Typy uczestników}

Drużyna wprowadza trzy typy uczestników, każdy z własnymi obowiązkami i poziomem zaangażowania.

\subsubsection{Jajko (Gość)}

\textbf{Plusy:}
\begin{itemize}
    \item Dostęp do wewnętrznej grupy dla gości.
    \item Możliwość 
    \item Możliwość zapoznania się z drużyną bez zobowiązań, aby po poznaniu zostać Papugą.

\end{itemize}

\textbf{Minusy:}
\begin{itemize}
    \item Droższe treningi.
    \item Brak gwarancji miejsca na treningu i meczu.
    \item Brak możliwości głosu i udziału w decyzjach organizacyjnych.
\end{itemize}

\subsubsection{Papuga (Zawodnik)}

\textbf{Plusy:}
\begin{itemize}
    \item Stałe składki z niższym kosztem.
    \item Dostęp do wewnętrznej grupy drużyny.
    \item Gwarancja miejsca na treningach oraz meczach.
    \item Dostęp do strojów treningowych.
    \item Systematyczna i regularna pomoc trenera oraz bardziej doświadczonych zawodników w poprawie umiejętności
    \item Drużynowy networking
    \item Wpływ na kierunek i rozwój drużyny
\end{itemize}

% TUTAJ WPISUJE

\end{document}