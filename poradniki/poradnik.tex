\documentclass{article}

\usepackage[polish]{babel}   % obsługa polskiego
\usepackage[utf8]{inputenc}  % kodowanie UTF-8
\usepackage[T1]{fontenc}     % kodowanie fontów, żeby polskie znaki były poprawne

\title{Zespół}
\date{\today}

\begin{document}

\maketitle

\section{Poradniki}

Poradniki obejmują wskazówki dotyczące techniki strzałów, jazdy na łyżwach, taktyki zespołowej oraz przygotowania fizycznego. Są przydatne zarówno dla początkujących, jak i doświadczonych zawodników.

\subsection{Spalony}

Spalony (ang. offside) występuje, gdy zawodnik znajdzie się w tercji obronnej przeciwnika zanim zostanie wbity tam krążek przez jego drużynę. Wówczas gra wraca do tercji środkowej (neutralnej). Obecnie stosuje się tzw. „tag-up offside”, czyli odłożony spalony. Oznacza to, że gracz znajdujący się na pozycji spalonej może z niej uciec, czyli wycofać się do strefy neutralnej, a następnie ponownie wjechać do strefy obronnej rywala – wówczas gra nie zostaje przerwana przez sędziego.

\subsection{Uwolnienie}

Uwolnienie (ang. icing) występuje, gdy gracz wybije krążek z własnej połowy (własnej tercji lub własnej połowy) na połowę przeciwnika w taki sposób, że krążek minie linię bramkową po drugiej stronie tafli oraz gdy żaden z zawodników nie będzie miał możliwości przejęcia go. Wówczas gra zostaje przerwana i następuje wznowienie w tercji obronnej drużyny dokonującej wystrzelenia krążka. Jest to tzw. zabronione uwolnienie, gdyż dokonywane jest przy równej liczbie graczy obu drużyn na lodzie. Uwolnienie dozwolone jest dopuszczalne dla drużyny grającej w liczebnym osłabieniu. Wówczas możliwość wystrzelenia krążka na drugi koniec tafli jest przywilejem w grze obronnej dla drużyny osłabionej. Ponadto zabronione uwolnienie jest bezskuteczne, gdy krążek trafi do bramki. Od sezonu 2008/09 IIHF wprowadziła nową zasadę, w myśl której drużyna przeciwna po dokonaniu zabronionego uwolnienia nie może dokonać zmiany zawodników i do wznowienia muszą przystąpić gracze, którzy przebywali na lodzie w momencie dokonania uwolnienia.

\end{document}